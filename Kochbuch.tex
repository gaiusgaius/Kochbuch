\input{/home/julius/dotfiles/preamble.tex}
\input{/home/julius/dotfiles/theorems_and_co.tex}
\usepackage{cuisine}

\title{Das Kochbuch um das Studentenleben zu überleben}
\author{Julius Stolz}
\date{Jul 31}

\begin{document}

\maketitle
%\tableofcontents

\begin{recipe}{Fisch mit Erbsenreis}{2 Portionen}{20 Minuten}
  \ingredient[1]{}{Haarnetz}
  \ingredient[\fr{1}{2}]{Tasse}{Reis}
  \ingredient[1]{Tasse}{Wasser}
  Zuerst den Reis im Haarnetz auswaschen und abtropfen lassen. Dann die halbe Tasse Reis gemeinsam mit einer Tasse Wasser in einem Topf mit Deckel erhitzen (Herd auf höchste Stufe). Wenn das Wasser kocht, den Herd ausschalten und den Deckel schräg auf den Topf legen.

  \ingredient[1]{}{Gusseisene Pfanne}
  \ingredient[4]{}{Dorschstücke}
  \ingredient[1]{Esslöffel}{Öl (Oliven oder Rapsöl)}
  \ingredient{nach belieben}{Salz}
  \ingredient{nach belieben}{Pfeffer}
  \ingredient{nach belieben}{Fischgewürz}
  \ingredient{nach belieben}{Zitronensaft}
  \ingredient[\fr12]{Esslöffel}{Knoblauch}
  \ingredient[1]{}{Knoblauchpresse}
  Man muss jetzt das ganze etwas mit dem Reis timen, dieser braucht nämlich länger. Vielleicht 5 Minuten nach dem man den Reis angefangen hat, jetzt mit dem Fisch beginnen.

  Zuerst die Pfanne erhitzen und Öl hinzugeben. Warten bis die Pfanne schön heiß wird (Hand über Pfanne halten, wenn sehr warm, dann richtig).
  Fische in die Pfanne. Salzen, Pfeffern. Das Fischgewürz und den Knoblauch dazu. Fische umdrehen und die Pfanne mit einem Deckel abdecken. Jetzt den Herd zurück drehen (auf 4). Nach einiger Zeit den Fisch wieder umdrehen.
  Tiefgekühlte Erbsen zum Reis dazu. Dieser muss vielleicht nochmal aufgekocht werden.
  
  Schlussendlich noch den Zitronensaft auf die Fische.
\end{recipe}

\begin{recipe}{Porridge mit Haferflocken}{1 Portion}{~3 Minuten}
  \ingredient{Nach Bedarf}{Haferlocken}
  \ingredient{Nach Bedarf}{Milch}
  \ingredient{Nach Bedarf}{Ahornsirup}
  Zuerst die Haferflocken in den Topf, dann Milch dazu und erhitzen. Einmal aufkochen lassen und dann mit Restwärme nach Bedarf kochen.
  Währenddessen Ahornsirup in die Schüssel. Dann alles in die Schüssel gießen.
\end{recipe}

\begin{recipe}{Reis mit Tomatensauce}{Mehrere Portionen}

\end{recipe}


\end{document}
