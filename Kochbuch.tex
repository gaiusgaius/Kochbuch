\documentclass{article}
\usepackage{cuisine}

\title{Das Kochbuch um das Studentenleben zu überleben}
\author{Julius Stolz}
\date{Jul 31}

\begin{document}

\maketitle
%\tableofcontents

\begin{recipe}{Fisch mit Erbsenreis}{2 Portionen}{20 Minuten}
  \ingredient[1]{}{Haarnetz}
  \ingredient[\fr{1}{2}]{Tasse}{Reis}
  \ingredient[1]{Tasse}{Wasser}
  Zuerst den Reis im Haarnetz auswaschen und abtropfen lassen. Dann die halbe Tasse Reis gemeinsam mit einer Tasse Wasser in einem Topf mit Deckel erhitzen (Herd auf höchste Stufe). Wenn das Wasser kocht, den Herd ausschalten und den Deckel schräg auf den Topf legen.

  \ingredient[1]{}{Gusseisene Pfanne}
  \ingredient[4]{}{Dorschstücke}
  \ingredient[1]{Esslöffel}{Öl (Oliven oder Rapsöl)}
  \ingredient{nach belieben}{Salz}
  \ingredient{nach belieben}{Pfeffer}
  \ingredient{nach belieben}{Fischgewürz}
  \ingredient{nach belieben}{Zitronensaft}
  \ingredient[\fr12]{Esslöffel}{Knoblauch}
  \ingredient[1]{}{Knoblauchpresse}
  Man muss jetzt das ganze etwas mit dem Reis timen, dieser braucht nämlich länger. Vielleicht 5 Minuten nach dem man den Reis angefangen hat, jetzt mit dem Fisch beginnen.

  Zuerst die Pfanne erhitzen und Öl hinzugeben. Warten bis die Pfanne schön heiß wird (Hand über Pfanne halten, wenn sehr warm, dann richtig).
  Fische in die Pfanne. Salzen, Pfeffern. Das Fischgewürz und den Knoblauch dazu. Fische umdrehen und die Pfanne mit einem Deckel abdecken. Jetzt den Herd zurück drehen (auf 4). Nach einiger Zeit den Fisch wieder umdrehen.
  Tiefgekühlte Erbsen zum Reis dazu. Dieser muss vielleicht nochmal aufgekocht werden.
  
  Schlussendlich noch den Zitronensaft auf die Fische.
\end{recipe}

\begin{recipe}{Porridge mit Haferflocken}{1 Portion}{~3 Minuten}
  \ingredient{Nach Bedarf}{Haferlocken}
  \ingredient{Nach Bedarf}{Milch}
  \ingredient{Nach Bedarf}{Ahornsirup}
  Zuerst die Haferflocken in den Topf, dann Milch dazu und erhitzen. Einmal aufkochen lassen und dann mit Restwärme nach Bedarf kochen.
  Währenddessen Ahornsirup in die Schüssel. Dann alles in die Schüssel gießen.
\end{recipe}

%\begin{recipe}{Reis mit Tomatensauce}{Mehrere Portionen}

\end{recipe}
\begin{recipe}{Mohnnudeln}{4 Portionen}{}

\ingredient{500g}{mehlige Kartoffeln}
\ingredient{100g}{Mehl}
\ingredient{50g}{Weizengrieß}
\ingredient{eine Prise}{Salz}
\ingredient{50g}{Butter}
\ingredient{1}{Dotter}
\ingredient{1 EL}{Butter (zum Anrösten in der Pfanne)}
\ingredient{100g}{Mohn}
\ingredient{80g}{Staubzucker}

\begin{enumerate}
\item Kartoffeln schälen, durch Kartoffelpresse drücken und mit Mehl, Butter, Grieß, einer Prise Salz und Dotter zu Teig kneten
\item Teig auf bemehlter Arbeitsfläche zu daumendicken Rollen formen,dies in 2cm dicke Scheiben schneiden und die dann zu Nudeln formen
\item Nudeln ca 5 Minuten köcheln bis sie an der Oberfläche schwimmen und bissfest sind, dann aus dem Wasser nehmen und abtropfen lassen
\item In einer Pfanne Butter zerlassen und Nudeln kurz darin anrösten
\item Mohn und Staubzucker vermischen und Nudeln im Gemisch wälzen 
\end{enumerate}
\end{recipe}

\begin{recipe}{Pancakes}{2 Portionen}{}

\ingredient{2}{Eier}
\ingredient{150g}{Mehl}
\ingredient{1/2 Packung}{Backpulver}
\ingredient{eine Prise}{Salz}
\ingredient{1 EL}{Staubzucker}
\ingredient{150 ml}{Milch}
\ingredient{0,5 EL}{ Butter oder ÖL (zum Anrösten in der Pfanne)}

\begin{enumerate}
\item Für die köstlichen Pancakes zuerst die Eier trennen. Danach das Eigelb, Mehl, Backpulver, Zucker, Salz und einen Schuss Milch in einer Schüssel zu einem zähflüssigen Teig verrühren.
\item Nun das Eiklar steif schlagen und unter die Teigmasse heben, sodass eine cremige Masse entsteht. Wenn der Teig zu fest ist, einfach noch ein wenig Milch hinzufügen
\item Dann Butter (oder Öl) in eine kleine Pfanne geben, mit einem Pinsel schön verteilen und heiß werden lassen. Den Teig portionsweise (mit einem Suppenschöpfer) in die Pfanne geben und auf jeder Seite 2 bis 3 Minuten goldbraun backen.
\item Traditionell serviert man die fertigen Pancakes mit flüssiger Butter und/oder Ahornsirup oder auch frischen Früchten.
\end{enumerate}
\end{recipe}

\begin{recipe}{Pizza}{eine Pizza}{}

\ingredient{125 ml}{Wasser}
\ingredient{250g}{Mehl}
\ingredient{21g}{Hefe (Germ)}
\ingredient{1/2 TL}{Salz}
\ingredient{1/2 Prise}{Zucker}
\ingredient{1 EL}{ÖL}

\begin{enumerate}
\item 120ml lauwarmes Wasser in einen Messbecher füllen. Hefe hineinbröseln und mit einer Prise Zucker und Salz verrühren. 10-15 Minuten gehen lassen.
\item Das Mehl in eine Schüssel geben. Flüssigkeit und Öl über das Mehl geben und mit den Knethaken des Handrührgeräts mindestens 5 Minuten kneten (von Hand mindestens 10 Minuten lang kneten). Zum Schluss mit den Händen noch einmal wenige Minuten weiterkneten, bis der Teig geschmeidig ist. Die Teigschüssel mit einem Tuch abdecken und an einem warmen Ort (ca. 35°C) etwa 40 Minuten gehen lassen.
\item Teig halbieren und auf bemehlter Arbeitsfläche  (Durchmesser ca. 28-30 cm)  ausrollen. Ofen auf 240 Grad (Umluft: 220) vorheizen. Ein Standardblech (38x45 cm) mit Backpapier belegen. Teige auf das Backblech legen und etwas in die Ränder zurechtdrücken. Jetzt den Pizzateig nach Belieben mit Tomatensoße und verschiedensten Zutaten belegen. Pizza im vorgeheizten Ofen etwa 15 Minuten backen.
\end{enumerate}
\end{recipe}

\end{document}
